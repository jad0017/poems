\documentclass{article}
\usepackage{fullpage}
\usepackage{enumitem}
\usepackage{tabularx}
\begin{document}

\begin{center}
{\LARGE Mondo-Mario Party}
\end{center}

\section{Preface}

Mario Party is one of the few, truly, timeless classics we have left in
this world.  The goal of this paper is to provide a framework for celebrating
the impressive achievements of the Nintendo corporation and the original
designer of the series, Shigeru Miyamoto.

\section{Terms}

\begin{description}[style=nextline]
\item[Bank Space] A yellow \textbf{space} depicting a coin that,
when passed, requires the \textbf{player} to deposit five (5) \textbf{coins}.
If the player does not have five (5) or more coins then all of the player's
coins are deposited.  If a player lands on the space then that player
receives all of the coins currently deposited in the bank.

\item[Blue Space] A \textbf{space} that, when landed on, grants the
\textbf{player} three (3) \textbf{coins}.

\item[Board] A collection of \textbf{spaces} and mechanics that the
\textbf{players} traverse.

\item[Board Event] A green ``?'' \textbf{space} that, when landed on,
initiates a Board Event.  A Board Event may be a \textbf{microgame},
forced movement, or alteration of the \textbf{board}.

\item[Bowser Space] A red \textbf{space} depicting fangs that,
when landed on, initiates a Bowser Time \textbf{microgame} (see
section \textbf{Microgames}).

\item[Chance Time] A green ``!'' \textbf{space} that, when landed on,
initiates the Chance Time \textbf{microgame} (see section
\textbf{Microgames}).

\item[Coin] Primary denomination of currency used on the \textbf{board}
and used to settle \textbf{ties} in the event that all \textbf{players}
have the same number of \textbf{stars}.

\item[Dice] A method of determining movement and order. The default
\textbf{die} is ten-sided (10-sided).  The die may be modified by the
use of \textbf{items}.

\item[Die] See \textbf{dice}.

\item[Duel Time] A green ``X'' \textbf{space} that, when landed on,
initiates the Duel Time \textbf{microgame} (see section \textbf{Microgames}).

\item[Item] An object usable by a \textbf{player} at the start of
his or her \textbf{turn} or during special events. See section \textbf{Items}).

\item[Item Store] A mushroom \textbf{space} that, when landed on,
allows the \textbf{player} to purchase one (1) \textbf{item} for a pre-defined
number of \textbf{coins}.

\item[Microgame] An overworld game or event that may occur on a \textbf{player}'s
\textbf{turn}. See section \textbf{Microgames}.

\item[Minigame] A game played by every \textbf{player} at the end of a \textbf{round}.
See section \textbf{Minigames}.  The winner(s) of a minigame are awarded
some amount of \textbf{coins}.

\item[Red Space] A \textbf{space} that, when landed on, takes up to
three (3) \textbf{coins} from the \textbf{player}.  If the player does
not have three (3) or more coins, then all of the player's remaining
coins are taken.

\item[Round] A round consists of \textbf{turns} and a \textbf{minigame}.
Every \textbf{player} must take one (1), and no more than one (1),
\textbf{turn} during a round.  Once all of the turns have been completed,
the minigame is initiated.

\item[Space] A circle of color, potentially with a symbol, that holds
special meaning when a \textbf{player} ends their \textbf{turn} on it.
These are walked on by the players.

\item[Star] The highest weighted denomination in the game.  The \textbf{player}
with the most stars at the end of the game is declared the \textbf{winner}
of Mondo-Mario Party.

\item[Star Space] A yellow \textbf{space} depicting a star that, when landed
on, allows the \textbf{player} to purchase a \textbf{star} for twenty (20)
\textbf{coins}.  The Star Space moves to a new location after a star has been
purchased.

\item[Tie] A tie occurs at the end of the game if more than one \textbf{player}
has the same number of \textbf{stars}.  Ties are resolved in the following order:
\begin{enumerate}
\item Stars
\item \textbf{Coins}
\item Bribes to the Princess
\item \textbf{Die} roll
\end{enumerate}

\item[Turn] Each \textbf{player} is given one (1) turn during each \textbf{round}.
A turn is characterized by the following set of ordered actions:
\begin{enumerate}
\item Use \textbf{item}, if available.
\item Roll either the ten-sided (10-sided) \textbf{die} or the result of
item use if the item modified the player's dice.
\item Move along the \textbf{board} until the rolled number of \textbf{spaces}
has been traversed.
\end{enumerate}

\item[Winner] The ultimate champion of Mondo-Mario Party.  There can only be one
and it probably wont be you.

\end{description}

\section{Items (Cards)}

Item Cards are similar to the way the traditional item system works except
they can be used to interfere with the board and other players.
Some Item Cards are placed on the board and effect 1) those who pass them, or 2)
those who land on them.  Cards are free, however they have an associated cost when
used.  They are given randomly by the Item Card Houses when passed.  The maximum
cards a player can hold is three (3).
\vspace{2ex}

\noindent
Cards and their effects are divided into four (4) categories.

\begin{description}[style=nextline]
\item[Movement]
These cards are colored \textbf{Green} and effect movement.

\item[Money]
These cards are colored \textbf{Yellow} and effect coinage.

\item[Attack]
These cards are colored \textbf{Red} and effect opponents.

\item[Board]
These cards are colored \textbf{Blue} and effect the board.

\item[Passive]
These cards are colored \textbf{Grey}, are held, and provide
a passive effect.
\end{description}

\noindent
Cards can have one of three designators.

\begin{description}[style=nextline]
\item[Board (B)]
These cards are placed on the board within a three (3) space span
of the player in either direction.  If accompanied by a lowercase
``d''---as in ``Bd''---then the card is removed from the board after use.

\item[Player (P)]
These cards are useable on the active player or an opponant.
If accompanied by a lowercase ``a''---as in ``Pa''---then the card is passive

\item[Both (S)]
These cards may either be placed on the board or used on either
the active player or an opponant. The lowercase ``d'' modifier is applicable.
\end{description}


\noindent
When placed on the board a special marker is used for visibility along with a
colored piece specifying which player placed the card.  This is important for
the effects of some cards (see below).

\subsection{Movement Cards}

These cards are \textbf{Green} and effect the movement of players.
\vspace{2ex}

\begin{tabular}{lccp{5cm}}
\textbf{Name} & \textbf{Cost} & \textbf{Designator} & \textbf{Effect} \\
\hline
Mushroom & 5 & P & Roll 2x10 dice \\
&&& \\
Poison Mushroom & 5 & P & Roll a 1x5 die \\
&&& \\
Stork & 5 & Bd & Lander moves back to the start space \\
&&& \\
Bubble & 10 & Sd & Move 10 spaces forward and lose all cards \\
&&& \\
Golden Mushroom & 10 & P & Roll 3x10 dice \\
&&& \\
Klepto & 10 & Sd & Move to the space of a random opponant \\
&&& \\
Metal Mushroom & 10 & P & Prevent the effect of traps for this turn \\
&&& \\
Warp Pipe & 10 & Sd & Swap places with a random opponant \\
&&& \\
Wiggler & 20 & P & Wiggle on down to the star space \\
&&& \\
\end{tabular}

\subsection{Money Cards}

These cards are \textbf{Yellow} and effect coinage.
\vspace*{2ex}

\begin{tabular}{lccl}
\textbf{Name} & \textbf{Cost} & \textbf{Designator} & \textbf{Effect} \\
\hline
Coin Block & 5 & P & Gain three (3) coins for every space moved \\
&&& \\
Spring & 5 & Bd & Lander loses ten (10) coins \\
&&& \\
Goomba & 10 & Bd & Lander gives 1x10 die of coins to the owner \\
&&& \\
Podoboo & 10 & Bd & Passing player loses ten (10) coins \\
&&& \\
Piranha Plant & 15 & Bd & Lander gives the owner half of their coins \\
&&& \\
Zap & 15 & Bd & Passer loses five (5) coins for each remaning space \\
\end{tabular}

\subsection{Attack Cards}

These cards are \textbf{Red} and effect opponents.
\vspace*{2ex}

\begin{tabular}{lccl}
\textbf{Name} & \textbf{Cost} & \textbf{Designator} & \textbf{Effect} \\
\hline
Toady & 5 & Bd & Owner steals a card from the lander \\
&&& \\
Blizard & 10 & Bd & Lander loses all of their held cards \\
&&& \\
Bomb-omb & 10 & Bd & Passer's remaining movement is halved \\
&&& \\
Kamek & 10 & Bd & Owner takes 1-3 cards of the lander from the board \\
&&& \\
Koopa Troopa & 10 & Bd & Passing player switches places with the owner \\
&&& \\
Lakitu & 10 & P & Steal a card from an opponant of your choosing \\
&&& \\
Magikoopa & 10 & Sd & Swap all cards with a random opponant \\
&&& \\
Thwomp & 10 & Bd & Passing player must stop moving \\
&&& \\
Twister & 10 & Sd & All player cards are shuffled and redistributed \\
&&& \\
Pink Boo & 30 & Bd & Lander loses a star, or twenty 20 coins if none \\
\end{tabular}

\subsection{Board Cards}

These cards are \textbf{Blue} and effect the board.
\vspace*{2ex}

\begin{tabular}{lccl}
\textbf{Name} & \textbf{Cost} & \textbf{Designator} & \textbf{Effect} \\
\hline
Bowser & 0 & BS & Bowser places a new Bowser Space on the board \\
&&& \\
Bank & 10 & B & Place a Bank Space \\
&&& \\
Boo & 15 & Sd & Summon Boo or place a Boo space \\
&&& \\
Chance & 20 & Sd & Initiate Chance Time or place a Chance Time Space \\
&&& \\
Duel & 20 & Sd & Initiate a Duel or place a Duel Space \\
&&& \\
Tweester & 15 & P & Reshuffle the Star Space \\
\end{tabular}

\subsection{Passive Cards}

These cards are \textbf{Grey} and are held and provide a passive effect.
Passive cards may be discarded at the start of your turn.
\vspace*{2ex}

\begin{tabular}{lccl}
\textbf{Name} & \textbf{Cost} & \textbf{Designator} & \textbf{Effect} \\
\hline
Boo-Away & 0 & Pa & Prevent Boo if attacked \\
&&& \\
Miracle & 0 & Pa & Collect three (3) and you gain a star \\
\end{tabular}

\section{Microgames}

Microgames are smaller games and events that happen throughout the overworld external
to Minigames.  They are divided by source.

\subsection{Board Event}

None so far. These are triggered during movement and affect the board.

\subsection{Bowser Time}

Bowser Time is initiated when a player lands on a Bowser Space.  The player
is confronted with the big baddy himself and is subjected to a wheel of
misfortune containing the following entries:

\begin{description}[style=nextline]
\item[Bowser Chance Time] See \textbf{Bowser Chance Time} section.
\item[Bowser Minigame] See \textbf{Bowser Minigames} section.

\item[Steal Star] Bowser takes one of the player's stars.  If the player has no
stars then bowser steals either twenty (20) coins or all of the player's coins,
whichever is smaller.

\item[Steal 5 Coins] Bowser takes five (5) coins from the player.
\item[Steal 10 Coins] Bowser takes ten (10) coins from the player.
\item[Steal 20 Coins] Bowser takes twenty (20) coins from the player.

\item[Socialist Party] Bowser takes all the coins from every player and
redistributes the wealth evenly.

\item[Tipsy Timmy] All players are under the effect of either the
poison or the reverse mushroom.

\item[The Pipes] All player positions are shuffled and redistributed.
\end{description}

\subsubsection{Bowser Chance Time}

The Chance Time Microgame is performed by Bowser with the target being the player.
The transfer die is modified such that player may never gain coins or stars.
See the \textbf{Chance Time} section.

\subsubsection{Bowser Minigame}

The player is subjected to a one-player minigame against bowser.  Bowser takes up to
twenty (20) coins from the player if they lose.
\vspace*{2ex}

\noindent
No games have been defined so far.

\subsection{Chance Time}

Chance Time allows a player to steal coins or stars from another player but can also
provide the opposite.


\noindent
Chance Time consists of two dice (decks of cards actually):

\begin{description}[style=nextline]
\item[Select Player] Roll/draw to determine the target player.
\item[Transfer] Roll/draw to determine what action is performed.
\item[Direction] Which direction the transfer is performed
(player-to-target or target-to-player).
\end{description}
\vspace*{2ex}

\noindent
Possible transfer actions (weights have yet to be determined):

\begin{description}[style=nextline]
\item[Give 5 Coins] Give the target (directional) 5 coins.
\item[Give 10 Coins] Give the target (directional) 10 coins.
\item[Give 20 Coins] Give the target (directional) 20 coins.
\item[Give 50 Coins] Give the target (directional) 50 coins.
\item[Give 1 Star] Give the target (directional) 1 star.
\item[Give 2 Stars] Give the target (directional) 2 stars.
\item[Swap Stars] Swaps stars with the target.
\item[Swap Coins] Swap coins with the target.
\item[Swap All] Swap both coins and stars with the target.
\end{description}

\subsection{Duel Time}

Duel Time allows a player to bet with another player that they can win.
Each player is deducted some amount of coins (drawn from a deck / die roll).
The players duel.
The winner takes the pooled coins.

\subsubsection{Duel Time Minigames}

\begin{description}[style=nextline]
\item[Mario Feud] - Single round, Family Feud-style
\item[...] - Rock paper scissors? we need more.
\end{description}

\section{Minigames}

Minigames are played at the end of each round.  They are divided up into multiple
categories: 1-vs-all, teams, and free-for-all.  Minigames award ten (10) coins
to the winning entity.
\vspace*{2ex}

\noindent
In Mario Party, the average breakdown of minigames per category (rounded up) is as follows.
\vspace*{2ex}

\begin{tabular}{l|c}
\textbf{Type} & \textbf{Count} \\
\hline
Free-For-All & 22 \\
1-Vs-3 & 11 \\
2-Vs-2 & 10 \\
Battle & 6 \\
\end{tabular}

\subsection{1-Vs-All}

All players of the winning team gain ten (10) coins.

\begin{description}[style=nextline]
\item[...]
\end{description}

\subsection{Teams}

All players of the winning team gain ten (10) coins.

\begin{description}[style=nextline]
\item[Codenames] A single round of codenames (themed?)
\item[...]
\end{description}

\subsection{Free-For-All}

Only one winner.  The winner takes the ten (10) coins.

\begin{description}[style=nextline]
\item[Coup] A single round of Mario-themed Coup (or normal coup depending on remaining funds).
\item[Boggled] A bigger Big Boggle.  Single round. (or normal big boggle depending on remaining funds).
\item[...]
\end{description}

\section{Required Materials}

Materials needed for the game.  The goal is to cut the costs as much as possible.

\begin{description}[style=nextline]
\item[Board Spaces] Poster board and spray paint?

\item[Coins] A couple decks of cards are cheap and easily made.
Probably have 1, 3, 5, 10, 20, 50 as values.

\item[Stars] Not sure yet, giving out cards for these just feels wrong. Might be able to find
some decently cheap pewter/zinc stars or I can pull the sheriff badges off the rest of the
cowboy hats that are currently sitting in my room.

\item[Items] These will be cards with descriptions on them.

\item[Minigames] Also cards! Shuffle and draw instead of making a giant-ass wheel.

\item[Other Random Stuff] Probably going to use decks of cards for almost all random
chance rolls since they're so cheap.  Like \$1.66 per 18 cards (\$5 for a deck of 54 cards).

\item[Player Designation] You all gonna be colors.  I original wanted to do Poles to be placed on the spaces,
but maybe colored bandannas?  Everyone has to wear their color and then a pair is placed on the space when
moving off of it (minigames/events).

\item[Dice] It's really hard to find big dice.  I've got some larger-than-normal foam dice but massive
dice don't exist outside of d6, so I think little dice will be needed here... The other option is switching
to cards instead of the dice also.  Figuring out the distributions is not difficult.  Simple probabilities.
\end{description}

\section{What's Next}

The following is a list of things that are necessary before the plan can be put into action:

\begin{itemize}
\item Finalize minigame lists (bowser, duel, 1-vs-all, free-for-all, team)
\item Determine max players (10? 12?)
\item Determine turn count (15 or less depending on players and how much time being taken?)
\item Design a map
\item Make card artwork
\item Pick an exact date
\item Select and acquire a trophy.
\item Find a location to use for the event
\end{itemize}
\vspace*{2ex}

\noindent
I'm hoping everything comes out cheap enough to not need to ask for admission or donations.

\end{document}
