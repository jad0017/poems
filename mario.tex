\documentclass{article}
\usepackage{fullpage}
\begin{document}

\begin{center}
{\LARGE Mondo-Mario Party}
\end{center}

\section{Preface}

Mario Party is one of the few, truly, timeless classics we have left in
this world.  The goal of this paper is to provide a framework for celebrating
the impressive achievements of the Nintendo corporation and the original
designer of the series, Shigeru Miyamoto.

\section{Terms}

\begin{description}
\item[Bank Space] A yellow \textbf{space} depicting a coin that,
when passed, requires the \textbf{player} to deposit five (5) \textbf{coins}.
If the player does not have five (5) or more coins then all of the player's
coins are deposited.  If a player lands on the space then that player
receives all of the coins currently deposited in the bank.

\item[Blue Space] A \textbf{space} that, when landed on, grants the
\textbf{player} three (3) \textbf{coins}.

\item[Board] A collection of \textbf{spaces} and mechanics that the
\textbf{players} traverse.

\item[Board Event] A green ``?'' \textbf{space} that, when landed on,
initiates a Board Event.  A Board Event may be a \textbf{microgame},
forced movement, or alteration of the \textbf{board}.

\item[Bowser Space] A red \textbf{space} depicting fangs that,
when landed on, initiates a Bowser Time \textbf{microgame} (see
section \textbf{Microgames}).

\item[Chance Time] A green ``!'' \textbf{space} that, when landed on,
initiates the Chance Time \textbf{microgame} (see section
\textbf{Microgames}).

\item[Coin] Primary denomination of currency used on the \textbf{board}
and used to settle \textbf{ties} in the event that all \textbf{players}
have the same number of \textbf{stars}.

\item[Dice] A method of determining movement and order. The default
\textbf{die} is ten-sided (10-sided).  The die may be modified by the
use of \textbf{items}.

\item[Die] See \textbf{dice}.

\item[Duel Time] A green ``X'' \textbf{space} that, when landed on,
initiates the Duel Time \textbf{microgame} (see section \textbf{Microgames}).

\item[Item] An object usable by a \textbf{player} at the start of
his or her \textbf{turn} or during special events. See section \textbf{Items}).

\item[Item Store] A mushroom \textbf{space} that, when landed on,
allows the \textbf{player} to purchase one (1) \textbf{item} for a pre-defined
number of \textbf{coins}.

\item[Microgame] An overworld game or event that may occur on a \textbf{player}'s
\textbf{turn}. See section \textbf{Microgames}.

\item[Minigame] A game played by every \textbf{player} at the end of a \textbf{round}.
See section \textbf{Minigames}.  The winner(s) of a minigame are awarded
some amount of \textbf{coins}.

\item[Red Space] A \textbf{space} that, when landed on, takes up to
three (3) \textbf{coins} from the \textbf{player}.  If the player does
not have three (3) or more coins, then all of the player's remaining
coins are taken.

\item[Round] A round consists of \textbf{turns} and a \textbf{minigame}.
Every \textbf{player} must take one (1), and no more than one (1),
\textbf{turn} during a round.  Once all of the turns have been completed,
the minigame is initiated.

\item[Space] A circle of color, potentially with a symbol, that holds
special meaning when a \textbf{player} ends their \textbf{turn} on it.
These are walked on by the players.

\item[Star] The highest weighted denomination in the game.  The \textbf{player}
with the most stars at the end of the game is declared the \textbf{winner}
of Mondo-Mario Party.

\item[Star Space] A yellow \textbf{space} depicting a star that, when landed
on, allows the \textbf{player} to purchase a \textbf{star} for twenty (20)
\textbf{coins}.  The Star Space moves to a new location after a star has been
purchased.

\item[Tie] A tie occurs at the end of the game if more than one \textbf{player}
has the same number of \textbf{stars}.  Ties are resolved in the following order:
\begin{enumerate}
\item Stars
\item \textbf{Coins}
\item Bribes to the Princess
\item \textbf{Die} roll
\end{enumerate}

\item[Turn] Each \textbf{player} is given one (1) turn during each \textbf{round}.
A turn is characterized by the following set of ordered actions:
\begin{enumerate}
\item Use \textbf{item}, if available.
\item Roll either the ten-sided (10-sided) \textbf{die} or the result of
item use if the item modified the player's dice.
\item Move along the \textbf{board} until the rolled number of \textbf{spaces}
has been traversed.
\end{enumerate}

\item[Winner] The ultimate champion of Mondo-Mario Party.  There can only be one
and it probably wont be you.

\end{description}

\section{Items}

The following are the items available during the game and their associated
costs.

\subsection{Item Prices}
\begin{tabular}{|l|c|}
\hline
\multicolumn{2}{|c|}{\textbf{Item Prices}} \\
\hline
\textbf{Item} & \textbf{Price (coins)} \\
\hline
Mushroom & 5 \\ % 2 dice blocks
Poison Mushroom & 5 \\ % Only allow 1, 2, 3, use on anyone
Reverse Mushroom & 5 \\ % Allows go backwards use on anyone
Warp Block & 5 \\ % switch with random player position
Dueling Glove & 10 \\ % Duel anyone in a duel microgame
Golden Mushroom & 10 \\ % 3 dice blocks
Lucky Lamp & 10 \\ % Force-moves the star
Plunder Chest & 10 \\ % steal random item from a player
Magic Lamp & 20 \\ % Moves the player to the star
%Boo Repellent & 10 \\ % stops boo
%Boo Bell & 15 \\ Summons boo
%Skeleton Key & 5 \\ open special doors
%Item Bag & 30 \\ 3 random items (+rares)
%Wacky Watch & N/A \\ advance to Last Five Turns (can extend or shorten)
%Barder Box  & N/A \\ swap all items with another player
%Koopa Kard & N/A \\ take all coins from bank when passing
%Lucky Charm & N/A \\ force a player to a Game Guy Microgame
\hline
\end{tabular}

\subsection{Item Descriptions}
\begin{description}
\item[Mushroom] The player rolls two (2) ten-sided (10-sided) dice instead
of one (1).

\item[Poison Mushroom] Used on another player (including the current player).
The next roll by the target may only be one (1), two (2), or three (3).

\item[Reverse Mushroom] Used on another player (including the current player).
The next roll by the target a six-sided (6-sided) die that moves the player backwards.

\item[Warp Block] Swap the position of the current player with another random player.

\item[Dueling Glove] Used on another player. Initiate a Duel Time Microgame with the target.
See section \textbf{Microgames}.

\item[Golden Mushroom] The player rolls three (3) ten-sided (10-sided) dice instead
of one (1).

\item[Lucky Lamp] Forces the star space to be moved.

\item[Plunder Chest] Used on another player. Steal a random item from the target.

\item[Magic Lamp] Moves the player to the star and allows the player to purchase it.
\end{description}

\section{Microgames}

Microgames are smaller games and events that happen throughout the overworld external
to Minigames.  They are divided by source.

\subsection{Board Event}

None so far. These are triggered during movement and effect the board.

\subsection{Bowser Time}

Bowser Time is initiated when a player lands on a Bowser Space.  The player
is confronted with the big baddy himself and is subjected to a wheel of
misfortune containing the following entries:

\begin{description}
\item[Bowser Chance Time] See \textbf{Bowser Chance Time} section.
\item[Bowser Minigame] See \textbf{Bowser Minigames} section.

\item[Steal Star] Bowser takes one of the player's stars.  If the player has no
stars then bowser steals either twenty (20) coins or all of the player's coins,
whichever is smaller.

\item[Steal 5 Coins] Bowser takes five (5) coins from the player.
\item[Steal 10 Coins] Bowser takes ten (10) coins from the player.
\item[Steal 20 Coins] Bowser takes twenty (20) coins from the player.

\item[Socialist Party] Bowser all the coins from every player and
redistributes the wealth evenly.

\item[Tipsy Timmy] All players are under the effect of either the
poison or the reverse mushroom.

\item[The Pipes] All players positions are shuffled and redistributed.
\end{description}

\subsubsection{Bowser Chance Time}

The Chance Time Microgame is performed by Bowser with the target being the player.
The transfer die is modified such that player may never gain coins or stars.
See the \textbf{Chance Time} section.

\subsubsection{Bowser Minigame}

The player is subjected to a one-player minigame against bowser.  Bowser takes up to
twenty (20) coins from the player if they lose.


\noindent
No games have been defined so far.

\subsection{Chance Time}

Chance Time allows a player to steal coins or stars from another player but can also
provide the opposite.


\noindent
Chance Time consists of two dice (decks of cards actually):

\begin{description}
\item[Select Player] Roll/draw to determine the target player.
\item[Transfer] Roll/draw to determine what action is performed.
\item[Direction] Which direction the transfer is performed
(player-to-target or target-to-player).
\end{description}


\noindent
Possible transfer actions (weights have yet to be determined):

\begin{description}
\item[Give 5 Coins] Give the target (directional) 5 coins.
\item[Give 10 Coins] Give the target (directional) 10 coins.
\item[Give 20 Coins] Give the target (directional) 20 coins.
\item[Give 50 Coins] Give the target (directional) 50 coins.
\item[Give 1 Star] Give the target (directional) 1 star.
\item[Give 2 Stars] Give the target (directional) 2 stars.
\item[Swap Stars] Swaps stars with the target.
\item[Swap Coins] Swap coins with the target.
\item[Swap All] Swap both coins and stars with the target.
\end{description}

\subsection{Duel Time}

Duel Time allows a player to bet with another player that they can win.
Each player is deducted some amount of coins (drawn from a deck / die roll).
The players duel.
The winner takes the pooled coins.

\subsubsection{Duel Time Minigames}

\begin{description}
\item[Mario Feud] - Single round, Family Feud-style
\item[...] - Rock paper scissors? we need more.
\end{description}

\section{Minigames}

Minigames are played at the end of each round.  They are divided up into multiple
categories: 1-vs-all, teams, and free-for-all.  Minigames award twenty (20) coins
to the winning entity.

\subsection{1-Vs-All}

Bad for high-player counts since the 'All' get the twenty (20) coins distributed over them
but the one (1) gets a good payout.

\begin{description}
\item[...]
\end{description}

\subsection{Teams}

Also bad for high-player counts since the teams  get the twenty (20) coins distributed over them.

\begin{description}
\item[Codenames] A single round of codenames (themed?)
\item[...]
\end{description}

\subsection{Free-For-All}

Only one winner.  The winner takes the twenty (20) coins.

\begin{description}
\item[Coup] A single round of Mario-themed Coup (or normal coup depending on remaining funds).
\item[Boggled] A bigger Big Boggle.  Single round. (or normal big boggle depending on remaining funds).
\item[...]
\end{description}

\section{Required Materials}

Materials needed for the game.  The goal is to cut the costs as much as possible.

\begin{description}
\item[Board Spaces] Poster board and spray paint?

\item[Coins] A couple decks of cards are cheap and easily made.
Probably have 1, 3, 5, 10, 20, 50 as values.

\item[Stars] Not sure yet, giving out cards for these just feels wrong. Might be able to find
some decently cheap pewter/zinc stars or I can pull the sheriff badges off the rest of the
cowboy hats that are currently sitting in my room.

\item[Items] These will be cards with descriptions on them.

\item[Minigames] Also cards! Shuffle and draw instead of making a giant-ass wheel.

\item[Other Random Stuff] Probably going to use decks of cards for almost all random
chance rolls since they're so cheap.  Like \$1.66 per 18 cards (\$5 for a deck of 54 cards).

\item[Player Designation] You all gonna be colors.  I original wanted to do Poles to be placed on the spaces,
but maybe colored bandannas?  Everyone has to wear their color and then a pair is placed on the space when
moving off of it (minigames/events).

\item[Dice] It's really hard to find big dice.  I've got some larger-than-normal foam dice but massive
dice don't exist outside of d6, so I think little dice will be needed here... The other option is switching
to cards instead of the dice also.  Figuring out the distributions is not difficult.  Simple probabilities.
\end{description}

\section{What's Next}

The following is a list of things that are necessary before the plan can be put into action:

\begin{itemize}
\item Finalize minigame lists (bowser, duel, 1-vs-all, free-for-all, team)
\item Determine max players (10? 12?)
\item Determine turn count (15 or less depending on players and how much time being taken?)
\item Design a map
\item Make card artwork
\item Pick an exact date
\item Select and acquire a trophy.
\item Find a location to use for the event
\end{itemize}


I'm hoping everything comes out cheap enough to not need to ask for admission or donations.

\end{document}
