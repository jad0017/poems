\documentclass{article}
\usepackage{fullpage}
\usepackage{enumitem}
\usepackage{tabularx}
\begin{document}

\begin{center}
{\LARGE Mondo-Mario Party}
\end{center}

\section{Preface}

Mario Party is one of the few, truly, timeless classics we have left in
this world.  The goal of this paper is to provide a framework for celebrating
the impressive achievements of the Nintendo corporation and the original
designer of the series, Shigeru Miyamoto.

\section{Terms}

\begin{description}[style=nextline]
\item[Bank Space] A yellow \textbf{space} depicting a coin that,
when passed, requires the \textbf{player} to deposit five (5) \textbf{coins}.
If the player does not have five (5) or more coins then all of the player's
coins are deposited.  If a player lands on the space then that player
receives all of the coins currently deposited in the bank.

\item[Blue Space] A \textbf{space} that, when landed on, grants the
\textbf{player} three (3) \textbf{coins}.

\item[Board] A collection of \textbf{spaces} and mechanics that the
\textbf{players} traverse.

\item[Board Event] A green ``?'' \textbf{space} that, when landed on,
initiates a Board Event.  A Board Event may be a \textbf{microgame},
forced movement, or alteration of the \textbf{board}.

\item[Bowser Space] A red \textbf{space} depicting fangs that,
when landed on, initiates a Bowser Time \textbf{microgame} (see
section \textbf{Microgames}).

\item[Chance Time] A green ``!'' \textbf{space} that, when landed on,
initiates the Chance Time \textbf{microgame} (see section
\textbf{Microgames}).

\item[Coin] Primary denomination of currency used on the \textbf{board}
and used to settle \textbf{ties} in the event that all \textbf{players}
have the same number of \textbf{stars}.

\item[Dice] A method of determining movement and order. The default
\textbf{die} is ten-sided (10-sided).  The die may be modified by the
use of \textbf{items}.

\item[Die] See \textbf{dice}.

\item[Duel Time] A green ``X'' \textbf{space} that, when landed on,
initiates the Duel Time \textbf{microgame} (see section \textbf{Microgames}).

\item[Item] An card/object usable by a \textbf{player} at the start of
his or her \textbf{turn} or during special events. See section \textbf{Items}).

\item[Item Store] A mushroom \textbf{space} that, when landed on,
allows the \textbf{player} the following options.

	\begin{enumerate}
	\item To purchase one (1) of many ``stock'' \textbf{item} cards
		for a pre-defined number of \textbf{coins}.
	\item To receive one (1) free random \textbf{item} card.
	\item To purchase a grab-bag of three (3) random \textbf{item} cards
		for a pre-defined number of \textbf{coins}.
	\end{enumerate}

\item[Microgame] An overworld game or event that may occur on a \textbf{player}'s
\textbf{turn}. See section \textbf{Microgames}.

\item[Minigame] A game played by every \textbf{player} at the end of a \textbf{round}.
See section \textbf{Minigames}.  The winner(s) of a minigame are awarded
some amount of \textbf{coins}.

\item[Red Space] A \textbf{space} that, when landed on, takes up to
three (3) \textbf{coins} from the \textbf{player}.  If the player does
not have three (3) or more coins, then all of the player's remaining
coins are taken.

\item[Round] A round consists of \textbf{turns} and a \textbf{minigame}.
Every \textbf{player} must take one (1), and no more than one (1),
\textbf{turn} during a round.  Once all of the turns have been completed,
the minigame is initiated.

\item[Space] A circle of color, potentially with a symbol, that holds
special meaning when a \textbf{player} ends their \textbf{turn} on it.
These are walked on by the players.

\item[Star] The highest weighted denomination in the game.  The \textbf{player}
with the most stars at the end of the game is declared the \textbf{winner}
of Mondo-Mario Party.

\item[Star Space] A yellow \textbf{space} depicting a star that, when landed
on, allows the \textbf{player} to purchase a \textbf{star} for twenty (20)
\textbf{coins}.  The Star Space moves to a new location after a star has been
purchased.

\item[Tie] A tie occurs at the end of the game if more than one \textbf{player}
has the same number of \textbf{stars}.  Ties are resolved in the following order:
\begin{enumerate}
\item \textbf{Stars}
\item \textbf{Coins}
\item Bribes to the Princess
\item \textbf{Die} roll
\end{enumerate}

\item[Turn] Each \textbf{player} is given one (1) turn during each \textbf{round}.
A turn is characterized by the following set of ordered actions:
\begin{enumerate}
\item Use \textbf{item}, if available.
\item Roll either the ten-sided (10-sided) \textbf{die} or the result of
item use if the item modified the player's dice.
\item Move along the \textbf{board} until the rolled number of \textbf{spaces}
has been traversed.
\end{enumerate}

\item[Winner] The ultimate champion of Mondo-Mario Party.  There can only be one
and it probably wont be you.

\item[Ztar] The opposite of a \textbf{star}.  Currently, a ztar is a forced purchase
resulting in a loss of twenty (20) \textbf{coins}.  Later this may expand to the
reduction of a \textbf{player's} star count, depending on circumstance.

\end{description}

%%%%%%%%%%%%%%%%%%%%%%%%%%%%%%%%%%%%%%%%%%%%%%%%%
% FLOW
%%%%%%%%%%%%%%%%%%%%%%%%%%%%%%%%%%%%%%%%%%%%%%%%%
\section{Game Flow}

Each game of Mondo-Mario Party is roughly equivalent.  The number of
turns is flexible, however 10-15 turns is the expected maximum.

\begin{enumerate}
\item Each player rolls for turn-order.  The player with the highest value
roll is first with each remaning spot in descending value.

\item For each round
	\begin{enumerate}
	\item Each player takes his/her turn in turn-order
		\begin{enumerate}
		\item The player is prompted for options
		\item The player rolls the die/dice
		\item The player moves along the board
		\item The player suffers the consequences of the resulting square
		\end{enumerate}
	\item Once all turns are complete, a minigame is held.
	\end{enumerate}

\item Once the rounds are complete, the winner is determined by total stars
followed by coins and so on.
\end{enumerate}

%%%%%%%%%%%%%%%%%%%%%%%%%%%%%%%%%%%%%%%%%%%%%%%%%
% ACTIONS
%%%%%%%%%%%%%%%%%%%%%%%%%%%%%%%%%%%%%%%%%%%%%%%%%
\section{Actions}

Actions are a set of standard responses that each player may perform in various
situations.  This section details the basic actions which may be altered by
board theme.


\subsection{Start-of-Turn Player Actions}

These actions are performed by the player at the start of his/her turn before rolling.

\begin{description}[style=nextline]
\item[Roll] Perform no actions and roll the die to begin the turn.
\item[Use Item] Use one of the player's available items.
\end{description}

\subsection{Item Actions}

These actions are dependent on the type of item and only apply when an item is being used.

\begin{description}[style=nextline]
\item[Use on Self] If the item is usable on the player, do so.
\item[Use on Board] If the item is usable on the board, do so.
The range that the player may ``lob'' the item in-front-of or behind
them is specified on the card.
\item[Use on Player] If the item is usable on another player, do so.
\item[Cancel] Cancel the use of the selected item.
\end{description}

%%%%%%%%%%%%%%%%%%%%%%%%%%%%%%%%%%%%%%%%%%%%%%%%%
% ITEMS
%%%%%%%%%%%%%%%%%%%%%%%%%%%%%%%%%%%%%%%%%%%%%%%%%

\section{Items (Cards)}

Item Cards are similar to the way the traditional item system works except
they can be used to interfere with the board and other players.
Some Item Cards are placed on the board and effect 1) those who pass them, or 2)
those who land on them.


\noindent
Cards may be obtained in one of five (5) ways.

\begin{enumerate}
\item Obtained for free by passing an Item Card House.
(one (1) random card)
\item Bought from a grab-bag at an Item Card House for 15 coins.
(three (3) random cards)
\item Bought from an Item Card House.
(one (1) specific card)
\item Obtained from a Board Event.
(depends on the event)
\item Obtained from a Bowser Event.
(depends on the event)
\end{enumerate}

Many cards that are obtained for free have an associated cost when used.
The maximum cards a player can hold is three (3) and so purchasing a grab-bag
with a non-empty inventory will yield the player only the cards he/she can
carry.
\vspace{2ex}

\noindent
Cards and their effects are divided into four (4) categories.

\begin{description}[style=nextline]
\item[Movement]
These cards are colored \textbf{Green} and effect movement.

\item[Money]
These cards are colored \textbf{Yellow} and effect coinage.

\item[Attack]
These cards are colored \textbf{Red} and effect opponents.

\item[Board]
These cards are colored \textbf{Blue} and effect the board.

\item[Passive]
These cards are colored \textbf{Grey}, are held, and provide
a passive effect.
\end{description}

\noindent
Cards can have one of three (3) designators.

\begin{description}[style=nextline]
\item[Board (B)]
These cards are placed on the board within a three (3) space span
of the player in either direction.  If accompanied by a lowercase
``d''---as in ``Bd''---then the card is removed from the board after use.

\item[Player (P)]
These cards are useable on the active player or an opponant.
If accompanied by a lowercase ``a''---as in ``Pa''---then the card is passive

\item[Both (S)]
These cards may either be placed on the board or used on either
the active player or an opponant. The lowercase ``d'' modifier is applicable.
\end{description}

\noindent
Cards marked with a ``T'' or ``BB'' are purchaseable ``stock'' cards.
They do not cost anything to use, however the cost listed is the up-front purchase
price.  Cards marked with ``T'' are bought from a Toad Card House whereas cards
marked with ``BB'' are bought from a Baby Bowser card house.
\vspace*{2ex}


\noindent
When placed on the board a special marker is used for visibility along with a
colored piece specifying which player placed the card.  This is important for
the effects of some cards (see below).

\subsection{Movement Cards}

These cards are \textbf{Green} and effect the movement of players.
\vspace{2ex}

\begin{tabular}{clccp{5cm}}
\textbf{S} & \textbf{Name} & \textbf{Cost} & \textbf{Designator}
& \textbf{Effect} \\
\hline
30 & Mushroom & 5 & P-T & Roll 2x10 dice \\
&&&& \\
10 & Poison Mushroom & 5 & P-BB & Roll a 1x5 die \\
&&&& \\
5 & Stork & 5 & Bd & Lander moves back to the start space \\
&&&& \\
5 & Bubble & 10 & Sd & Move 10 spaces forward and lose all cards \\
&&&& \\
15 & Golden Mushroom & 10 & P-T & Roll 3x10 dice \\
&&&& \\
5 & Klepto & 10 & Sd & Move to the space of a random opponant \\
&&&& \\
10 & Metal Mushroom & 10 & P-BB & Prevent the effect of traps for this turn \\
&&&& \\
15 & Warp Pipe & 10 & Sd-BB & Swap places with a random opponant \\
&&&& \\
10 & Wiggler & 20 & P-T & Wiggle on down to the star space \\
&&&& \\
10 & Reverse Shroom & 5 & P-BB & Roll 1x10 but move backwards \\
\end{tabular}

\subsection{Money Cards}

These cards are \textbf{Yellow} and effect coinage.
\vspace*{2ex}

\begin{tabular}{clccl}
\textbf{S} & \textbf{Name} & \textbf{Cost} & \textbf{Designator}
& \textbf{Effect} \\
\hline
5 & Coin Block & 5 & P & Gain three (3) coins for every space moved \\
&&&& \\
10 & Spring & 5 & Bd & Lander loses ten (10) coins \\
&&&& \\
5 & Goomba & 10 & Bd & Lander gives 1x10 die of coins to the owner \\
&&&& \\
10 & Podoboo & 10 & Bd & Passing player loses ten (10) coins \\
&&&& \\
5 & Piranha Plant & 15 & Bd & Lander gives the owner half of their coins \\
&&&& \\
5 & Zap & 15 & Bd & Passer loses five (5) coins for each remaning space \\
\end{tabular}

\subsection{Attack Cards}

These cards are \textbf{Red} and effect opponents.
\vspace*{2ex}

\begin{tabular}{clccl}
\textbf{S} & \textbf{Name} & \textbf{Cost} & \textbf{Designator} & \textbf{Effect} \\
\hline
10 & Toady & 5 & Bd & Owner steals a card from the lander \\
&&&& \\
5 & Blizard & 10 & Bd & Lander loses all of their held cards \\
&&&& \\
5 & Bomb-omb & 10 & Bd & Passer's remaining movement is halved \\
&&&& \\
5 & Kamek & 10 & Bd & Owner takes 1-3 cards of the lander from the board \\
&&&& \\
10 & Koopa Troopa & 10 & Bd & Passing player switches places with the owner \\
&&&& \\
5 & Lakitu & 10 & P & Steal a card from an opponant of your choosing \\
&&&& \\
10 & Magikoopa & 10 & Sd & Swap all cards with a random opponant \\
&&&& \\
10 & Thwomp & 10 & Bd & Passing player must stop moving \\
&&&& \\
5 & Twister & 10 & Sd & All player cards are shuffled and redistributed \\
&&&& \\
5 & Pink Boo & 30 & Bd & Lander loses a star, or twenty 20 coins if none \\
\end{tabular}

\subsection{Board Cards}

These cards are \textbf{Blue} and effect the board.
\vspace*{2ex}

\begin{tabular}{clccl}
\textbf{S} & \textbf{Name} & \textbf{Cost} & \textbf{Designator} & \textbf{Effect} \\
\hline
5 & Bowser & 0 & BS & Bowser places a new Bowser Space on the board \\
&&&& \\
2 & Bank & 10 & B & Place a Bank Space \\
&&&& \\
2 & Boo & 15 & Sd & Summon Boo or place a Boo space \\
&&&& \\
2 & Chance & 20 & Sd & Initiate Chance Time or place a Chance Time Space \\
&&&& \\
2 & Duel & 20 & Sd & Initiate a Duel or place a Duel Space \\
&&&& \\
10 & Tweester & 15 & P-BB & Reshuffle the Star Space \\
\end{tabular}

\subsection{Passive Cards}

These cards are \textbf{purple} and are held and provide a passive effect.
Passive cards may be discarded at the start of your turn.
\vspace*{2ex}

\begin{tabular}{clccl}
\textbf{S} & \textbf{Name} & \textbf{Cost} & \textbf{Designator} & \textbf{Effect} \\
\hline
10 & Boo-Away & 0 & Pa & Prevent Boo if attacked \\ % -BB?
&&&& \\
15 & Miracle & 0 & Pa & Collect three (3) and you gain a star \\
\end{tabular}

\subsection{Board Cards}

These cards are \textbf{Orange} and are used in various ways depending on the
theme of the board.
\vspace*{2ex}

\begin{tabular}{clccl}
\textbf{S} & \textbf{Name} & \textbf{Cost} & \textbf{Designator} & \textbf{Effect} \\
\hline
20 & Basic & N/A & N/A & Do something basic. \\
&&&& \\
20 & Special & N/A & N/A & Do something not-so-basic. \\
\end{tabular}

%%%%%%%%%%%%%%%%%%%%%%%%%%%%%%%%%%%%%%%%%%%%%%%%%%%%%%
% Themes
%%%%%%%%%%%%%%%%%%%%%%%%%%%%%%%%%%%%%%%%%%%%%%%%%%%%%%

\section{Board Themes}

Board Themes provide an extra layer of mechanics in an attempt to make the overworld
portion of the game more interesting.

\subsection{Murder Mystery}

Oh deary, deary, you've all been invited to a very posh party but it seems something
has gone awry.  One of the guests has be muuuuuuurdered and it's up to you to solve
this unsightly affair before one of the other, less desirable, guests, steals the
spot light.
\vspace*{2ex}


\noindent
This theme takes the form of a ``Logic Puzzle'' (one of them grid thingies).  How it
works:

\begin{itemize}
\item A new option ``Search For Clue'' is available at the start of a player's turn
in place of ``Use Item''.  As In, if you use an item you cannot search for a clue.
There is a 30-50\% chance of finding one.

\item A new option ``Solve Mystery'' is available at the start of a player's turn
in place of ``Use Item''.  When performed the player must provide their solution.

\item The ``Basic Board Cards'' are used passivly and provide a guaranteed clue
on Search. Only 10 will be present in-game. %Only 15 will be present in-game.

%\item The ``Special Board Cards'' are used passivly and provide two clues at once.
%Only five (5) will be present in-game.
\end{itemize}

\noindent
The first player to solve the puzzle gains a star.

\subsection{Monster Mash}

You monster.  Rawr.  You smash good.  Building crumble to ground.  Most destroy
get stawr.


\noindent
``Buildings'' will be present around the board.  They take three (3) hits to smash.
Once a building is completely destroyed, a player identifier of the monster who
smashed the building will be placed in its spot.  The player who has caused
the most mayham at the end of the game will gain a star.  How it works:

\begin{itemize}
\item When passing by a non-destroyed building, player's have the option of
``Taking a Swing'', dealing one hit to the building.

\item If the building crumbles, the player's identifier is left on the spot
for accountability.

\item The ``Basic Board Cards'' are used passivly and provide a double-swing
on a building.  Fifteen (15) will be present in-game.

\item The ``Special Board Cards'' are used passivly and ressurrect a fallen
building.  A building that has been resurrected is no longer associated with
the original smasher and is free to be claimed by the next player who deals
the final blow.  Ten (10) will be present in-game.
\end{itemize}

\subsection{Which Witch is Which?}

No deets yet but I want a witchcraft theme... something with scrying orbs or
tarot cards?

\section{Microgames}

Microgames are smaller games and events that happen throughout the overworld external
to Minigames.  They are divided by source.

\subsection{Bank Space}

While not technically a microgame, the bank is an important aspect of the game.
When passed, a player must give the bank five (5) coins.  When landed on,
the player gains the number of coins currently kept in the bank.  The assumption
being that the player robs the bank.

\subsection{Board Event}

None so far. These are triggered during movement and affect the board.

\subsection{Bowser Time (2 spaces + 5 cards)}

Bowser Time is initiated when a player lands on a Bowser Space.  The player
is confronted with the big baddy himself and is subjected to a wheel of
misfortune containing the following entries:

\begin{description}[style=nextline]
\item[Bowser Chance Time] See \textbf{Bowser Chance Time} section.
\item[Bowser Minigame] See \textbf{Bowser Minigames} section.

\item[Steal Star] Bowser takes one of the player's stars.  If the player has no
stars then bowser steals either twenty (20) coins or all of the player's coins,
whichever is smaller.

\item[Steal 5 Coins] Bowser takes five (5) coins from the player.
\item[Steal 10 Coins] Bowser takes ten (10) coins from the player.
\item[Steal 20 Coins] Bowser takes twenty (20) coins from the player.

\item[Socialist Party] Bowser takes all the coins from every player and
redistributes the wealth evenly.

\item[Tipsy Timmy] All players are under the effect of either the
poison or the reverse mushroom.

\item[The Pipes] All player positions are shuffled and redistributed.

\item[Zat Zis Unfortunaz] Add a Ztar hidden as a Star to the board and shuffle the
two Star spaces.
\end{description}

\subsubsection{Bowser Chance Time}

The Chance Time Microgame is performed by Bowser with the target being the player.
The transfer die is modified such that player may never gain coins or stars.
See the \textbf{Chance Time} section.

\subsubsection{Bowser Minigame}

The player is subjected to a one-player minigame against bowser.  Bowser takes up to
twenty (20) coins from the player if they lose.
\vspace*{2ex}

\noindent
No games have been defined so far.

\subsection{Chance Time (1 space + 2 cards)}

Chance Time allows a player to steal coins or stars from another player but can also
provide the opposite.


\noindent
Chance Time consists of two dice (decks of cards actually):

\begin{description}[style=nextline]
\item[Select Player] Roll/draw to determine the target player.
\item[Transfer] Roll/draw to determine what action is performed.
\item[Direction] Which direction the transfer is performed
(player-to-target or target-to-player).
\end{description}
\vspace*{2ex}

\noindent
Possible transfer actions (weights have yet to be determined):

\begin{description}[style=nextline]
\item[Give 5 Coins] Give the target (directional) 5 coins.
\item[Give 10 Coins] Give the target (directional) 10 coins.
\item[Give 20 Coins] Give the target (directional) 20 coins.
\item[Give 50 Coins] Give the target (directional) 50 coins.
\item[Give 1 Star] Give the target (directional) 1 star.
\item[Give 2 Stars] Give the target (directional) 2 stars.
\item[Swap Stars] Swaps stars with the target.
\item[Swap Coins] Swap coins with the target.
\item[Swap All] Swap both coins and stars with the target.
\end{description}

\subsection{Duel Time (1 space + 2 cards)}

Duel Time allows a player to bet with another player that they can win.
Each player is deducted some amount of coins (drawn from a deck / die roll).
The players duel.
The winner takes the pooled coins.

\subsubsection{Duel Time Minigames}

\begin{description}[style=nextline]
\item[Mario Feud] - Single round, Family Feud-style
\item[...] - Rock paper scissors? we need more.
\end{description}

\section{Minigames}

Minigames are played at the end of each round.  They are divided up into multiple
categories: 1-vs-all, teams, and free-for-all.  Minigames award ten (10) coins
to the winning entity.
\vspace*{2ex}

\noindent
In Mario Party, the average breakdown of minigames per category (rounded up) is as follows.
\vspace*{2ex}

\begin{tabular}{l|c}
\textbf{Type} & \textbf{Count} \\
\hline
Free-For-All & 22 \\
1-Vs-3 & 11 \\
2-Vs-2 & 10 \\
Battle & 6 \\
\end{tabular}

\subsection{1-Vs-All}

All players of the winning team gain ten (10) coins.

\begin{description}[style=nextline]
\item[...]
\end{description}

\subsection{Teams}

All players of the winning team gain ten (10) coins.

\begin{description}[style=nextline]
\item[Codenames] A single round of codenames (themed?)
\item[...]
\end{description}

\subsection{Free-For-All}

Only one winner.  The winner takes the ten (10) coins.

\begin{description}[style=nextline]
\item[Coup] A single round of Mario-themed Coup (or normal coup depending on remaining funds).
\item[Boggled] A bigger Big Boggle.  Single round. (or normal big boggle depending on remaining funds).
\item[...]
\end{description}

\section{Required Materials}

Materials needed for the game.  The goal is to cut the costs as much as possible.
The entries marked with Blizzy the Snowperson \textbf{(*\_*)} have already been obtained.

\begin{description}[style=nextline]
\item[(*\_*) Board Spaces] Using colored cones meant for setting up field boundaries in
soccer or gym class or whatever.  They only come in five (5) colors so additions to the
spaces will have to be made to differntiate all of the different types.  There is a hole
on the top of each which might be usable to hold some other form of identifying info.

\item[Coins] A couple decks of cards are cheap and easily made.
Probably have 1, 3, 5, 10, 20, 50 as values.

\item[(*\_*) Dice] I got three large demonstration-sized foam D10s.

\item[Items] These will be cards with descriptions on them.

\item[Item Designators] Somehow we need to mark on the board what type the item
occupying it is and who owns it.  Currently I'm thinking the card used goes on the
space (somehow) face-down so only the type is known along with a player designator.

\item[Minigames] Also cards! Shuffle and draw instead of making a giant-ass wheel.

\item[Other Random Stuff] Probably going to use decks of cards for almost all random
chance rolls since they're so cheap.  Like \$1.66 per 18 cards (\$5 for a deck of 54 cards).

\item[Player Designation] You all gonna be colors.  I original wanted to do Poles to be placed on the spaces,
but maybe colored bandannas?  Everyone has to wear their color and then a pair is placed on the space when
moving off of it (minigames/events).

\item[(*\_*) Stars] Using a bunch of plastic star-coins.

\end{description}

\section{What's Next}

The following is a list of things that are necessary before the plan can be put into action:

\begin{itemize}
\item Finalize minigame lists (bowser, duel, 1-vs-all, free-for-all, team)
\item Determine max players (Groups of 4, can probs run more than one at once.)
\item Determine turn count (10-15 or less depending on players and how much time being taken?)
\item Design a map
\item Make card artwork
\item Pick an exact date
\item Select and acquire a trophy.
\item Find a location to use for the event
\end{itemize}
\vspace*{2ex}

\noindent
I'm hoping everything comes out cheap enough to not need to ask for admission or donations.

\end{document}
