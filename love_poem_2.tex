\documentclass{article}
\usepackage{fullpage}
\begin{document}

\begin{center}
\textbf{Princess Writes A Love Poem 2} \\
{\small\textbf{Untitled}} \\
\vspace*{2ex}
\begin{tabular}{l}
Slowly \\
render \\
the years \\
where the \\
good never fades \\
slip me \\
into your shade \\
hold me tight \\
breathe me life \\
call me lilac \\
on summer nights \\
let me fall with you \\
let me fall with you \\
% Lilac is a very positive flower that is used as a symbol of rebirth
% or vitality which is why it's use in this stanza is so important...
% it conveys the appropriate meaning without being in your face about it...
% that is to say a relationship can rejuvenate a person and then the
% "fall with you" is both a sentiment and a biblical reference which
% worked (in my mind) as inevitability from a birth and letting the world
% imprint on the couple, or more accurately to say the couple will
% re-experience the world together and grow out of a happy innocence into
% something more experienced.
\\
This tortoise \\
was lost to the rain \\
home was so far away \\
you led me inside \\
where I could leave my shell \\
you let me inside \\
where I no longer fell \\
I dried over the years \\
warm smiles enveloped \\
ear to ear \\
coursing your heat \\
into litany \\
% I hate this stanza... oh gosh golly does it need some love.  The rhyming
% is just atrocious and naive.  I used "Litany" loosely in this piece.
% Litany is a clerical term meant for the dry repetition of some ritual.
% The use here is simply that the narrator has grown accustom to the new
% feelings or whatever.  This stanza was meant to represent finding a new
% home but I missed my mark.  Oh and the rain part of the stanza is meant
% to tie back to "summer nights" as well.
\\
Slowly \\
render \\
the years \\
where we meet \\
and long \\
to meet again \\
hold me lucid \\
by your side \\
with ends \\
long forgotten \\
breathe with me \\
a brighter dream \\
endlessly \\
% The litany as repitition in the previous stanza is why this one starts the
% same as the first.  Sooo Lucid has very strong connotations towards an
% awareness during a period where you aren't normally super aware (lucid
% dreams, or when a person is mentally deficient later on in life due to
% any number of reasons).  I'm also using this kinda loosely.  It's meant
% here as an awakening for the narrator in that the other body brings out
% a better side of him/her.  The "ends long forgotten" also doesn't
% necessarily make any sense as written... it was definitely a
% "completing the circle" reference but the surrounding context was lost
% in revision.
\end{tabular}
\end{center}

\end{document}
